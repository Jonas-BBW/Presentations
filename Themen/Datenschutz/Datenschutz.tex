% Schriftgröße = 9 Punkt, Dokumentenklasse für Präsentationen
\documentclass[9pt]{beamer}
%~~~~~~~~~~~~~~~~~~~~~~~~~~~~~~~~~~~~~~~~~~~~~~~~~~~~~~~~~~~~~~~~~~~~~~~~~~~~~~
% Standardpakete
\usepackage[utf8]{inputenc}
\usepackage[T1]{fontenc}
\usepackage[ngerman]{babel}
\usepackage[autostyle=true,german=quotes]{csquotes}
\usepackage[sfdefault]{roboto}
%~~~~~~~~~~~~~~~~~~~~~~~~~~~~~~~~~~~~~~~~~~~~~~~~~~~~~~~~~~~~~~~~~~~~~~~~~~~~~~

%~~~~~~~~~~~~~~~~~~~~~~~~~~~~~~~~~~~~~~~~~~~~~~~~~~~~~~~~~~~~~~~~~~~~~~~~~~~~~~
% zusätzliche Pakete
\usepackage{hyperref}
%\usepackage{amsmath}       % mathematische Formeln
%\usepackage{tikz}          % Zeichnungen
%\usepackage{booktabs}      % für schönere Tabellen
%~~~~~~~~~~~~~~~~~~~~~~~~~~~~~~~~~~~~~~~~~~~~~~~~~~~~~~~~~~~~~~~~~~~~~~~~~~~~~~

%~~~~~~~~~~~~~~~~~~~~~~~~~~~~~~~~~~~~~~~~~~~~~~~~~~~~~~~~~~~~~~~~~~~~~~~~~~~~~~
% selbsterstellte Befehle
\newcommand{\tabitem}{~~\llap{\textbullet}~~}       % Stichpunkte in Tabellen
\newcommand{\tabspace}{~~\llap{\vphantom{ } }~~}    % Abstand für weitere Zeilen eines Stichpunktes
%~~~~~~~~~~~~~~~~~~~~~~~~~~~~~~~~~~~~~~~~~~~~~~~~~~~~~~~~~~~~~~~~~~~~~~~~~~~~~~

%~~~~~~~~~~~~~~~~~~~~~~~~~~~~~~~~~~~~~~~~~~~~~~~~~~~~~~~~~~~~~~~~~~~~~~~~~~~~~~
\usepackage{styles/fluxmacros}
\usefolder{styles}
% Möglichkeiten: asphalt, blue, red, green, gray
\usetheme[style=asphalt]{flux}
% Der Logo-Befehl darf nicht auskommentiert werden, kann aber freigelassen werden
%~~~~~~~~~~~~~~~~~~~~~~~~~~~~~~~~~~~~~~~~~~~~~~~~~~~~~~~~~~~~~~~~~~~~~~~~~~~~~~

%~~~~~~~~~~~~~~~~~~~~~~~~~~~~~~~~~~~~~~~~~~~~~~~~~~~~~~~~~~~~~~~~~~~~~~~~~~~~~~
% Informationen
\title{DATENSCHUTZ}
\subtitle{}
\author{Jonas Schönbach, Christopher Walther und Marius Niesler}
\institute{SRH Berufsbildungswerk Dresden}
\date{\today}
%~~~~~~~~~~~~~~~~~~~~~~~~~~~~~~~~~~~~~~~~~~~~~~~~~~~~~~~~~~~~~~~~~~~~~~~~~~~~~~
\begin{document}
    \titlepage

    \begin{frame}
        \frametitle{Gliederung}
        \tableofcontents
    \end{frame}



    \section{Datenschutz allgemein}
        \subsection{Geschichte}
            \begin{frame}{Datenschutz allgemein}{}
                \centering\huge
                {
                   Datenschutz ist der Schutz sämtlicher\\Informationen,
                   die nicht\\für die Allgemeinheit frei verfügbar sind.
                }
            \end{frame}
            \begin{frame}{Datenschutz allgemein}{Geschichte}
            \begin{itemize}
                \item 1960er: Debatte um John F. Kennedy's Nationales Datenzentrum
                \begin{itemize}
                    \item scheiterte, gesetzliche Grundlage für die Verarbeitung \\personenbezogener Daten gefordert
                    \item Ergebnis: Verabschiedung des Privacy Act (1974)
                \end{itemize}
                \item in Deutschland wurde Übersetzung für Privacy gesucht
                \begin{itemize}
                    \item Begriff Datenschutz entstand (Anlehnung an vorhandenen Maschinenschutz)
                \end{itemize}
                \item 1970: Hessen schafft erstes Datenschutzgesetz weltweit
                \item 1977: Bundesdatenschutzgesetz entstand
            \end{itemize}
            \end{frame}


        \subsection{wichtige Grundsätze}
            \begin{frame}{Datenschutz allgemein}{wichtige Grundsätze}
                \begin{itemize}
                    \item Grundsatz der Datenvermeidung und Datensparsamkeit\\
                    \begin{itemize}
                        \item[$\rightarrow$] nur unbedingt erforderliche Daten sammeln
                    \end{itemize}
                    \item Erlaubnisvorbehalt
                    \begin{itemize}
                        % mathematische Aussagen kommen in '$'-Zeichen %
                        \item[$\rightarrow$] Nutzer muss Erhebung, Speicherung oder Verarbeitung seiner Daten
                        vorher zustimmen (sofern nicht gesetzlich angeordnet)
                        \item[$\rightarrow$] Erhebung von Daten gegen den Willen des Nutzers / ohne dessen Kenntnis ist nicht zulässig
                    \end{itemize}
                    \item Trennungsgrundsatz
                        \begin{itemize}
                            \item[$\rightarrow$] mehrere Diensteanbieter dürfen gesammelte Informationen nicht zusammenfügen
                            \item[$\rightarrow$] Austausch von Daten darf Datensparsamkeit nicht umgehen
                            \item[$\rightarrow$] Nutzer hat Auskunftsansprüche gegenüber Stellen, die seine Daten sammeln
                            \item[$\rightarrow$] Nutzer hat Anspruch auf Berichtigung gegenüber Stellen, die seine Daten sammeln bei Unvollständigkeit/Fehlerhaftigkeit
                        \end{itemize}
                \end{itemize}
            \end{frame}


        \subsection{Einzelangaben}
            \begin{frame}{Datenschutz allgemein}{Einzelangaben}
                \begin{itemize}
            		\item persönliche oder sachliche Verhältnisse einer Person,\\die ihr zugeordnet werden können\newline
            	\end{itemize}
            	\vspace{0.25cm}
                \centerline{\begin{minipage}{280pt}
    		        \begin{tabular}{| l | l |}
            		    \hline
            		    \textbf{persönliche Verhältnisse} & \textbf{sachliche Verhältnisse}\\
                        \hline
            		    \tabitem Name           &   \tabitem Kraftfahrzeugnummer\\
            		    \tabitem Familienstand  &   \tabitem Kfz-Kennzeichen\\
            		    \tabitem Geburtsdatum   &   \tabitem Personalausweisnummer\\
            		    \tabitem Alter          &   \tabitem Sozialversicherungsnummer\\
            		    \tabitem Anschrift      &   \tabitem Vorstrafen\\
            		    \tabitem Telefonnummer  &   \tabitem genetische Daten und Krankendaten\\
                        \tabitem E-Mail Adresse &   \\
                        \hline
                    \end{tabular}
                \end{minipage}}
            \end{frame}



    \section{Personenbezogene Daten}
        \subsection{Bestimmte Personen}
            \begin{frame}{Personenbezogene Daten}{}
                \begin{itemize}
        		    \item sind all jene Informationen, die...\\...sich auf eine natürliche Person beziehen\\...auf eine natürliche Person beziehbar sind\\und so Rückschlüsse auf deren Persönlichkeit erlauben.\newline{}
        		    \item \textbf{laut DSGVO besonders schützenswert:}\\genetische, biometrische und Gesundheitsdaten, rassische und ethnische Herkunft, politische Meinungen, religiöse oder weltanschauliche Überzeugungen, Gewerkschaftszugehörigkeit
        	    \end{itemize}
            \end{frame}

            \begin{frame}{Personenbezogene Daten}{Bestimmte Personen}
                \begin{itemize}
            		\item eine Person wird als bestimmte Person definiert, wenn die Daten mit der betroffenen Person verbunden sind
            		\item Beispiele: Muttermale, Fingerabdruck, Augenfarben, Haarfarbe, Kfz-Kennzeichen, Kontonummer, Personalausweisnummer etc.
            		\item die Daten können mit Personen verbunden sein
            		\item aus dem Inhalt wird ein Bezug zu der Person hergestellt
            	\end{itemize}
            \end{frame}


        \subsection{Bestimmbare Personen}
            \begin{frame}{Personenbezogene Daten}{Bestimmbare Personen}
                \begin{itemize}
            		\item eine Person wird als bestimmbar definiert, wenn sie direkt oder indirekt identifiziert wird
            		\item die Identität einer Person wurde mit Hilfe von Zusatzwissen identifiziert
            	\end{itemize}
            \end{frame}


        \subsection{Natürliche Personen}
            \begin{frame}{Personenbezogene Daten}{Natürliche Personen}
                \begin{itemize}
            		\item sind Menschen unserer Bevölkerung mit der Funktion als Rechtssubjekt
            		\item ist unabhängig von Geschlecht, Herkunft und Staatsangehörigkeit
            		\item kann nicht von behördlichen oder gerichtlichen Entscheidungen aberkannt werden
            		\item Träger kann sich auch nicht durch Verzichtserklärungen aufheben oder beschränken
            		\item sind Träger von Rechten und Pflichten und damit rechtsfähig
            		\item sie endet mit dem Tod, endgültiger Ausfall des Gehirns
            		\item neben Rechtsfähigkeit und Nichtrechtsfähigkeit gibt es Teilrechtsfähigkeit
            		    \begin{itemize}
                            \item gilt für Nasciturus (ungeborener Mensch)
                        \end{itemize}
            	\end{itemize}
            \end{frame}


        \subsection{Juristische Personen}        
            \begin{frame}{Personenbezogene Daten}{Juristische Personen}
                \begin{itemize}
            		\item sind Vereinigungen von Personen oder Sachen
            		\item rechtlich geregelte Einheit
            		\item bekamen von der Rechtsordnung die Rechtsfähigkeit
            		\item sind Träger von eigenen Rechten und Pflichten
            		\item man unterscheidet zwischen Privatrecht und öffentlichem Recht
            	\end{itemize}
            \end{frame}

        \begin{frame}{Personenbezogene Daten}{Juristische Personen - Körperschaften}
                \begin{itemize}
            		\item definiert ein Zusammenschluss von Personen
            		\item Zusammenarbeit auf ein geminsames, nicht individuell, einzelnes Ziel hin
            		\item Mitglieder können wechseln
            		\item das angestrebte und erklärte Ziel bleibt gleich
            		\item Körperschaften sind im juristischen Sinne immer eine juristische Person
            	\end{itemize}
            \end{frame}

        \begin{frame}{Personenbezogene Daten}{Juristische Personen - privates Recht}
                \begin{itemize}
            		\item sind in der Regel Körperschaften gemeint
            		\item unabhängig von den Mitgliedern
            		\item Mehrheitsprinzip
            		\item Haftung durch Gesellschaftsvermögen
            		\item Beispiele: GmbH, AG, e.G., Vereine
            	\end{itemize}
            \end{frame}

            \begin{frame}{Personenbezogene Daten}{Juristische Personen - öffentliches Recht}
                    \begin{itemize}
                		\item in der Regel ebenso Körperschaften
                		\item sind Bund, Länder und Gemeinden
                		\item können auch öffentliche Stiftungen, Anstalten und Sparkassen sein
                		\item Beliehene gelten auch als juristische Personen
                	\end{itemize}
                \end{frame}



    \section{Rechtsgrundlagen und Rechte}
        \subsection{BDSG}
            \begin{frame}{Rechtsgrundlagen und Rechte}{Bundesdatenschutzgesetz}
                \begin{itemize}
                    \item bildet Datenschutzrahmen in Deutschland
            		\item regelt mit den Datenschutzgesetzen der Länder den Umgang mit persönlichen Daten in Deutschland
            		\item Originalfassung vom 27.01.1977
            		\item letzte Neufassung vom 30.06.2017
            		\item Inkrafttreten der Neufassung am 25.05.2018
            	\end{itemize}
            \end{frame}


        \subsection{DSGVO}
            \begin{frame}{Rechtsgrundlagen und Rechte}{Datenschutzgrundverordnung}
                \begin{itemize}
	                \item EU-weite Verordnung zum Schutz von persönlichen Daten
	                \item bildet Datenschutzrahmen in der EU
            		\item Inkrafttreten: 24.05.2016
            		\item muss seit dem 25.05.2018 angewendet werden
            	\end{itemize}
            \end{frame}
        

        \subsection{Daten mit erhöhtem Schutz}
            \begin{frame}{Rechtsgrundlagen und Rechte}{Daten mit erhöhtem Schutz}
                \begin{itemize}
                    \item Art.9 DSGVO
                \end{itemize}
                \vspace{0.25cm}
                \centerline{\begin{minipage}{340pt}
                        \begin{tabular}{| l | l |}
                        \hline
                        \textbf{Geschützte Daten} & \textbf{Ausnahmen}\\
                        \hline
                        \tabitem rassische und ethnische Herkunft       & \tabitem  durch ausdrückliche Zustimmung\\
                        \tabitem politische Meinungen                   & \tabspace (bis auf Ausnahmen)\\
                        \tabitem religiöse/philosophische Überzeugungen & \tabitem  zur Ausübung von \\
                        \tabitem Gewerkschaftszugehörigkeit             & \tabspace Rechten und Pflichten\\
                        \tabitem Gesundheit                             & \tabspace für lebenswichtige Interessen,\\
                        \tabitem sexuelle Orientierung                  & \tabspace wenn Person unfähig\\
                                                                        & \tabitem  für Daten, die offensichtlich sind\\
                                                                        & \tabitem  für Gesundheitsvorsorge\\
                        \hline
                    \end{tabular}    
                \end{minipage}}
            \end{frame}


        \subsection{Widerrufsrecht}
            \begin{frame}{Rechtsgrundlagen und Rechte}{Widerrufsrecht}
                \begin{itemize}
                    \item Art.21 DSGVO
            		\item Recht auf Widerspruch, wenn Daten zur Direktwerbung genutzt werden
            		\item wenn widersprochen wird, dürfen Daten nicht mehr verarbeitet werden
            		\item Person muss spätestens zum Zeitpunkt der ersten Kommunikation hingewiesen werden
            	\end{itemize}
            \end{frame}


        \subsection{Recht auf Einsicht}
            \begin{frame}{Rechtsgrundlagen und Rechte}{Auskunftsrecht (Recht auf Einsicht)}
                \begin{itemize}
                \item Art.15 DSGVO
            		\item Verarbeitungszwecke
            		\item Kategorien der verarbeiteten Daten
            		\item (Kriterien für) Speicherdauer
            		\item Empfänger der Daten
            		\item Kopie der zu verarbeitenden Daten muss gegeben werden
            	\end{itemize}
            \end{frame}


        \subsection{Löschungsrecht}
            \begin{frame}{Rechtsgrundlagen und Rechte}{Löschungsrecht}
                \begin{itemize}
                    \item auch Recht auf Vergessen genannt 
                    \item Art. 17 DSGVO
                    \item Gründe zur Löschung von Daten
                    \begin{itemize}
                        \item Daten sind nicht mehr notwendig
                        \item Person widerruft Einwilligung
                        \item unrechtmäßige Verarbeitung der Daten
                        \item nicht gültig zur Ausübung der freien Meinungsäußerung
                        \item zur Erfüllung einer rechtlichen Verpflichtung
                    \end{itemize}
                \end{itemize}
            \end{frame}


        \subsection{Berichtigungsrecht}
            \begin{frame}{Rechtsgrundlagen und Rechte}{Berichtigungsrecht}
                \begin{itemize}
                    \item Art. 16 DSGVO
                    \item Personen dürfen jederzeit die Berichtigung/ die Vervollständigung von falschen/fehlenden Daten anfordern
                \end{itemize}
        \end{frame}



    \section{Skandale}
        \subsection{Google Street View}
            \begin{frame}{Skandale}{Google Street View}
                \begin{itemize}
            		\item Mai 2007 führte Google \enquote{Street View} ein
            		\item Google hat seitdem immer wieder Beschwerden von Datenschützern
            		\item zeigt bspw. Menschen die Erotik-Geschäfte betraten oder verließen
            		\item Gegenmaßnahme: Gesichter und Kennzeichen mit Unschärfe-Effekte unkenntlich zu machen
            	\end{itemize}
            \end{frame}


        \subsection{Facebook und Cambridge Analytica}
            \begin{frame}{Skandale}{Facebook und Cambridge Analytica}
                \begin{itemize}
            		\item hat bewiesen: Facebook kann Daten nicht schützen
            		\begin{itemize}
            		    \item Warum:
            		    \begin{itemize}
            		        \item[--] App-Entwickler dürfen persönlichen Daten von Nutzern zu erfassen
            		        \item[--] Daten von Freunden können auch ausgelesen werden
            		        \item[--] zehntausende derartiger Fälle dadurch denkbar
            		    \end{itemize}
            		\end{itemize}
            		\vspace{0.25cm}
            		\item[ ]Was hat das nun mit Cambridge Analytica zu tun?
            		\begin{itemize}
            		    \item  Aleksandr Kogan entwickelte Persönlichkeitstest  "thisisyourdigitallife"
            		    \item 270 000 zu meist US-amerikanische Nutzer machten mit
            		    \item erfasste zusätzlich Daten von Millionen ihrer ahnungslosen Freunde
            		    \item gab Daten an Cambridge Analytica weiter
            		    \item Facebook wusste Bescheid,\\kümmerte sich nicht darum Daten löschen zu lassen
            		    \item Resultat: 500.000£-Geldstrafe für Facebook
            		\end{itemize}
            	\end{itemize}
            \end{frame}


        \subsection{Sonys Spyware CDs}
            \begin{frame}{Skandale}{Sonys Spyware CDs}
                \begin{itemize}
            		\item Anti-Piraterie-Maßnahme "XCP"
            		\item beim Abspielen einer Sony CD installierte XCP heimlich eine versteckte Rootkit-Software
            		\item Software sendet Informationen von CD und der IP vom Rechner an Sony
            		\item Spyware machte den Rechner anfälliger gegen Trojaner und Viren
            		\item Sammelklagen gingen gegen den Konzern ein
            		\item Kartellbehörde entschied das Sony jedem 100€ Schadensersatzzahlung zukommen soll
            	\end{itemize}
            \end{frame}



    \section{Quellen}
        \begin{frame}{Quellen}{}
            \begin{itemize}
                \item \href{https://dsgvo-gesetz.de/}{https://dsgvo-gesetz.de/}
                \item \href{https://www.gesetze-im-internet.de/tmg/}{https://www.gesetze-im-internet.de/tmg/}
                \item \href{https://www.gesetze-im-internet.de/bdsg/}{https://www.gesetze-im-internet.de/bdsg/}
                \item \href{https://www.wbs-law.de/it-recht/datenschutzrecht/was-ist-datenschutz/}{https://www.wbs-law.de/it-recht/datenschutzrecht/was-ist-datenschutz/}
                \item \href{https://www.juraforum.de/lexikon/natuerliche-person}{https://www.juraforum.de/lexikon/natuerliche-person}
                \item \href{https://www.juraforum.de/lexikon/juristische-person}{https://www.juraforum.de/lexikon/juristische-person}
                \item \href{https://www.juraforum.de/lexikon/koerperschafthttps://www.juraforum.de/lexikon/koerperschaft}{https://www.juraforum.de/lexikon/koerperschaft}
                \item \href{https://www.itsystemkaufmann.de/natuerliche-und-juristische-personen-der-unterschied/}{https://www.itsystemkaufmann.de/natuerliche-und-juristische-personen-der-unterschied/}
                \item \href{https://datenschutz-agentur.de/alle-informationen-ueber-eine-bestimmte-oder-bestimmbare-natuerliche-person-personenbezogen/}{https://datenschutz-agentur.de/alle-informationen-ueber-eine-bestimmte-oder-bestimmbare-natuerliche-person-personenbezogen/}
                %\item \href{}{}
            \end{itemize}
        \end{frame}


        \begin{frame}[plain]
            \begin{center}
                \huge\textbf{Vielen Dank für Ihre Aufmerksamkeit!}
            \end{center}
        \end{frame}
    
\end{document}