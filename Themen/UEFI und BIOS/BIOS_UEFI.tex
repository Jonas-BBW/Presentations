% Schriftgröße = 9 Punkt, Dokumentenklasse für Präsentationen
\documentclass[9pt]{beamer}
%~~~~~~~~~~~~~~~~~~~~~~~~~~~~~~~~~~~~~~~~~~~~~~~~~~~~~~~~~~~~~~~~~~~~~~~~~~~~~~
% Standardpakete
\usepackage[utf8]{inputenc}
\usepackage[T1]{fontenc}
\usepackage[ngerman]{babel}
\usepackage[autostyle=true,german=quotes]{csquotes}
%~~~~~~~~~~~~~~~~~~~~~~~~~~~~~~~~~~~~~~~~~~~~~~~~~~~~~~~~~~~~~~~~~~~~~~~~~~~~~~

%~~~~~~~~~~~~~~~~~~~~~~~~~~~~~~~~~~~~~~~~~~~~~~~~~~~~~~~~~~~~~~~~~~~~~~~~~~~~~~
% zusätzliche Pakete
\usepackage[sfdefault]{roboto}
\usepackage{hyperref}
%~~~~~~~~~~~~~~~~~~~~~~~~~~~~~~~~~~~~~~~~~~~~~~~~~~~~~~~~~~~~~~~~~~~~~~~~~~~~~~

%~~~~~~~~~~~~~~~~~~~~~~~~~~~~~~~~~~~~~~~~~~~~~~~~~~~~~~~~~~~~~~~~~~~~~~~~~~~~~~
% selbsterstellte Befehle
\newcommand{\tabitem}{~~\llap{\textbullet}~~}       % Stichpunkte in Tabellen
\newcommand{\tabspace}{~~\llap{\vphantom{ } }~~}    % Abstand für weitere Zeilen eines Stichpunktes
%~~~~~~~~~~~~~~~~~~~~~~~~~~~~~~~~~~~~~~~~~~~~~~~~~~~~~~~~~~~~~~~~~~~~~~~~~~~~~~

%~~~~~~~~~~~~~~~~~~~~~~~~~~~~~~~~~~~~~~~~~~~~~~~~~~~~~~~~~~~~~~~~~~~~~~~~~~~~~~
\usepackage{styles/fluxmacros}
\usefolder{styles}
% Möglichkeiten: asphalt, blue, red, green, gray
\usetheme[style=asphalt]{flux}
%~~~~~~~~~~~~~~~~~~~~~~~~~~~~~~~~~~~~~~~~~~~~~~~~~~~~~~~~~~~~~~~~~~~~~~~~~~~~~~

%~~~~~~~~~~~~~~~~~~~~~~~~~~~~~~~~~~~~~~~~~~~~~~~~~~~~~~~~~~~~~~~~~~~~~~~~~~~~~~
% Informationen
\title{BIOS und UEFI}
\subtitle{}
\author{Jonas Schönbach, Christopher Walther und Marius Niesler}
\institute{SRH Berufsbildungswerk Dresden}
\date{\today}
%~~~~~~~~~~~~~~~~~~~~~~~~~~~~~~~~~~~~~~~~~~~~~~~~~~~~~~~~~~~~~~~~~~~~~~~~~~~~~~
\begin{document}
    \titlepage

    \begin{frame}
        \frametitle{Gliederung}
        \tableofcontents
    \end{frame}
    
    \section{UEFI vs. BIOS}
    \subsection{Basic Input/Output System}
        \begin{frame}{UEFI vs. BIOS}{BIOS}
            \begin{itemize}
                \item ist Schnittstelle zwischen Hardware und Betriebssystem
                \item führt Power-On Self-Test (POST) durch
                \item seit Jahren veraltet, hat nur 1MB Speicher
                \begin{itemize}
                    \item[$\rightarrow$] Starten vieler Peripheriegeräte schwierig
                \end{itemize}
                \item verwendet Master Boot Record
                \begin{itemize}
                    \item[$\rightarrow$] von Festplatten mit mehr als 2,2TB booten unmöglich
                \end{itemize}
                \item sehr limitierter Funktionsumfang
                \item startet ersten Boot Loader unter Berücksichtigung der Bootpriorität 
                \item Bedienung nur mit Tastatur
                \item bei Systemen mit 64bit-Architektur fast vollständig von UEFI abgelöst
            \end{itemize}
    \end{frame}
    
    \subsection{Unified Extensible Firmware Interface}
        \begin{frame}{UEFI vs. BIOS}{UEFI}
            \begin{itemize}
                \item verwendet „Globally Unique Identifier“-Partitionstabelle
                \item kann mit bis zu 9.4ZB umgehen,\\Größe des Internets wird auf ca. 3ZB geschätzt
                \item ist als kleines Betriebssystem zu verstehen\\(Browser, RAM-Test, … möglich)
                \item Bedienung mit Maus und Tastatur möglich
                \item aufwendige, graphische Benutzeroberfläche und hohe Auflösungen
                \item Updates unkompliziert
                \item Dateien befinden auf der Festplatte in Partition „/boot/EFI/“, Netzwerk, ...
                \item kein ROM verwendet
            \end{itemize}
        \end{frame}

    \section{Geschichte}
        \subsection{BIOS}
            \begin{frame}{Geschichte}{Entwicklung des BIOS}
                \begin{itemize}
                    \item erstes BIOS 1975 für Betriebssystem CP/M von Gary A. Kildall entwickelt
                	\item 1981 beginnt IBM mit dem ersten IBM-PC; BIOS wird weiterentwickelt
                	\item alte Betriebssysteme griffen auf BIOS zu, um angeschlossene Hardware anzusprechen
                	\item mittlerweile nur noch für Systemstart zuständig
                	\item BIOS ist auf einem EEPROM oder Flash-Speicher gespeichert
                	\begin{itemize}
                		\item wird auf Motherboard verlötet oder sitzt in einem Sockel
                	\end{itemize}
                \end{itemize}
            \end{frame}

            \begin{frame}{Geschichte}{Entwicklung des BIOS - Read Only Memory}
                \begin{itemize}
                    \item permanenter Halbleiterspeicher
                    \item Daten werden dauerhaft und meist unveränderlich gespeichert
                    \item enthalten Betriebssysteme, Anwendungsprogramme und Firmware
                    \item werden in der Regel nicht an die Platine gelötet
                    \begin{itemize}
                        \item[$\rightarrow$] meistens in Sockel gesteckt (Austauschbarkeit)
                    \end{itemize}
                    \item heute dienen Flash-Speicher als ROM Ersatz
                    \item Flash-Speicher ist überschreibbar, auch während des Betriebes
                    \item ROM mittlerweile nur verwendet wenn Änderung des Speicherinhalts unerwünscht
                \end{itemize}
            \end{frame}

        \subsection{UEFI}
            \begin{frame}{Geschichte}{UEFI}
                \begin{itemize}
                    \item 1998: Intel startet Entwicklung des \textbf{EFI} (\textbf{E}xtensible \textbf{F}irmware \textbf{I}nterface)
                    \item 2007: IT-Unternehmen wie Intel, AMD,  Apple, IBM, Lenovo, Microsoft entschieden \textbf{EFI} als einheitlichen „Nachfolger“ für das BIOS
                    \item Begriff \enquote{UEFI} (\textbf{U}nified \textbf{E}xtensible \textbf{F}irmware \textbf{I}nterface) entstand
                \end{itemize}
            \end{frame}

    \section{Verschiedenes}
    
        \subsection{UEFI Shell}
        \begin{frame}{Verschiedenes}{UEFI Shell}
            \begin{itemize}
                \item textbasierte Benutzerschnittstelle
                \item kann direkt vom UEFI anstelle des Betriebssystems gestartet werden
                \item ermöglicht Arbeiten in Ebene zwischen UEFI und Betriebssystem
                \begin{itemize}
                    \item[$\rightarrow$] das Erstellen und Ausführen von Skripten, die den Bootvorgang\\beeinflussen möglich
                    \item[$\rightarrow$] über Shell können Betriebssysteme installiert werden
                \end{itemize}
                
            \end{itemize}
        \end{frame}

    \subsection{Power-On-Self-Test}
        \begin{frame}{Verschiedenes}{Power-On-Self-Test}
        \begin{itemize}
            \item erstes Programm, das ein Computer ausführt
            \item überprüft grundlegende Komponenten des Computer auf Funktionsfähigkeit
        \end{itemize}
        \begin{exampleblock}{Ablauf}
            \begin{enumerate}
                \item Überprüfung der Funktionsfähigkeit der CPU\\
                \item Überprüfung der CPU-nahen Bausteine
                \item Überprüfung des CMOS-RAM
                \item Überprüfung des CPU-nahen Cache-Speichers
                \item Überprüfung der ersten 64 Kilobyte des Arbeitsspeichers
                \item Überprüfung des Grafik-Speichers und der Grafik-Ausgabe-Hardware/Software
                \item Grafikkarte wird in Betrieb genommen
                \item Überprüfung des restlichen Arbeitsspeichers 
                \item Überprüfung der Tastatur (PS/2-Schnittstelle)
                \item Überprüfung von weiterer Peripherie, u. a. Diskettenlaufwerke und Festplatten
            \end{enumerate}
        \end{exampleblock}
        \end{frame}
        
        \subsection{Compatibility Support Module}
        \begin{frame}{Verschiedenes}{Compatibility Support Module}
            \begin{itemize}
                \item benötigt für Unterstützung alter (32bit-)Betriebssysteme und alter Hardware
                \item emuliert BIOS
                \item ermöglicht Boot von MBR-partitionierten Festplatten
                \item Intel unterstützt CSM nur noch bis 2020 (Bekanntgabe: November 2017)
            \end{itemize}
        \end{frame}

    \subsection{CMOS}
    \begin{frame}{Verschiedenes}{CMOS}
        \begin{itemize}
            \item Abkürzung für \enquote{Complementary Metal Oxide Semiconductor} (komplementärer Metalloxidhalbleiter)
            \item extrem stromsparende Halbleitertechnologie
            \item ein SRAM mit dieser Technologie wird als CMOS-RAM bezeichnet
            \begin{itemize}
                \item Bestandteil vom jeden PC
                \item wird nach Abschaltung des Computers von Batterie mit Strom versorgt
                \begin{itemize}
                    \item[$\rightarrow$] kein Datenverlust solange Batterie funktionsfähig
                \end{itemize}
                \item speichert BIOS-Einstellungen
                \item enthält keinen direkt ausführbaren Code
            \end{itemize}
        \end{itemize}
    \end{frame}
    

    \begin{frame}{Verschiedenes}{CMOS Reset}
        \begin{itemize}
        	\item ist das vorübergehende Entfernen der Flachbatterie vom Mainboard
        	\item Daten werden dabei gelöscht
        	\item es gilt zu Beachten:
        	\begin{itemize}
        		\item Rechner in jedem Fall ausschalten
        		\item Rechner vom Stromnetz trennen
        		\item nach dem entfernen der Batterie 10 - 20 Minuten warten
        	\end{itemize}
        \end{itemize}
    \end{frame}

    \section{Quellen}
        \begin{frame}{Quellen}{}
            \begin{itemize}
                \item \href{https://de.wikipedia.org/wiki/Power-on_self-test}{de.wikipedia.org/wiki/Power-on\_{}self-test}
                \item \href{https://de.wikipedia.org/wiki/BIOS}{de.wikipedia.org/wiki/BIOS}
                \item \href{https://de.wikipedia.org/wiki/Unified_Extensible_Firmware_Interface}{de.wikipedia.org/wiki/Unified\_{}Extensible\_{}Firmware\_{}Interface}
                \item \href{https://phoenixts.com/blog/uefi-vs-legacy-bios/}{phoenixts.com/blog/uefi-vs-legacy-bios/}
                \item \href{https://wiki.ubuntuusers.de/EFI_Grundlagen/}{wiki.ubuntuusers.de/EFI\_{}Grundlagen/}
                \item \href{https://en.wikipedia.org/wiki/Unified_Extensible_Firmware_Interface\#CSM_booting}{en.wikipedia.org/wiki/Unified\_{}Extensible\_{}Firmware\_{}Interface\#{}CSM\_{}boot}
                \item \href{https://www.elektronik-kompendium.de/sites/com/0807131.htm}{https://www.elektronik-kompendium.de/sites/com/0807131.htm}
                \item \href{https://www.elektronik-kompendium.de/sites/com/0309181.htm}{https://www.elektronik-kompendium.de/sites/com/0309181.htm}
                \item \href{https://de.wikibooks.org/wiki/Computerhardware:_BIOS:_CMOS}{https://de.wikibooks.org/wiki/Computerhardware:\_{}BIOS:\_{}CMOS}
                \item \href{https://www.pcwelt.de/tipps/Systemeinstellung-So-setzen-Sie-das-Bios-richtig-zurueck-8136828.html}{https://www.pcwelt.de/tipps/Systemeinstellung-So-setzen-Sie-das-Bios-richtig-zurueck-8136828.html}
                %\item \href{}{}
            \end{itemize}
        \end{frame}
    \section{UEFI-Vorstellung}
    
    \begin{frame}[plain]
            \begin{center}
                \huge\textbf{Vielen Dank für Ihre Aufmerksamkeit!}
            \end{center}
        \end{frame}
\end{document}